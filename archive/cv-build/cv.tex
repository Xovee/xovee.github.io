\documentclass{article}
\usepackage[utf8]{inputenc}
\usepackage[scale=.75]{geometry}
\usepackage[full]{textcomp}
\usepackage[scaled,osf]{garamondx}
% \usepackage[lf,osf]{ebgaramond}
\usepackage{CJKutf8}
\usepackage{url}
\usepackage{xcolor}
\usepackage{enumitem}
\usepackage[
    hidelinks,
    pdfauthor={Xovee Xu},
    pdftitle={Curriculum Vitae},
]{hyperref}

\pagestyle{headings}
\markright{\textbf{Xovee Xu}}

\setlength\parindent{2em}


% \author{Xovee Xu}
% \date{}

\thispagestyle{empty}

\begin{document}

\begin{center}
    \Huge{
    \textbf{Xovee Xu}}
\end{center}

% \begin{center}
%     \Large
%     \textit{Curriculum Vitae}
% \end{center}


\noindent Master Student in Software Engineering\hfill Chinese Name: \begin{CJK*}{UTF8}{gkai}徐增\end{CJK*}

\noindent School of Information and Software Engineering\hfill xovee@live.com

\noindent University of Electronic Science and Technology of China\hfill \url{https://www.xovee.cn}

\noindent Shahe Campus, No.4, Sect.2, N Jianshe Rd, Chenghua Dist\hfill ORCID: \href{https://orcid.org/0000-0001-6415-7558}{0000-0001-6415-7558}

\noindent Chengdu, Sichuan 610054, China

\setlength{\parskip}{3pt}


\vspace{-8pt}
\section*{Education}
\vspace{-4pt}
\indent 

% Ph.D. in Computer Science (2021 - present)

B.S. and M.S. in Software Engineering (2014 - 2018 - present)

University of Electronic Science and Technology of China (UESTC), Chengdu, Sichuan, China

% \vspace{-8pt}
% \section*{Academic Employment}
% \vspace{-8pt}
% \indent

\vspace{-8pt}
\section*{Distinction}
\vspace{-4pt}
\indent

INFOCOM 2020 Student Conference Award (2020), IEEE ComSoc

\hangindent=3em Outstanding Graduate (2020), First Prize Scholarship (2020), University of Electronic Science and Technology of China (UESTC)

\vspace{-8pt}
\section*{Academic Service}
\vspace{-4pt}
% \indent

% \subsection*{\textit{Editorial Board}}

\subsection*{\textnormal{\textit{Reviewing}}}
\vspace{-4pt}
\indent


AAAI (1), AAAI Conference on Artificial Intelligence, 2021

BigData (1), IEEE International Conference on Big Data, 2020

KDD (1), ACM SIGKDD Conference on Knowledge Discovery and Data Mining, 2019

IoT-J (1), IEEE Internet of Things Journal, 2021

TKDE (1), IEEE Transactions on Knowledge and Data Engineering, 2021

TNNLS (1), IEEE Transactions on Neural Networks and Learning Systems, 2020

TON (1), IEEE/ACM Transactions on Networking, 2021

\subsection*{\textnormal{\textit{Membership}}}
\vspace{-4pt}
\indent 

ACM, Member (2020)

IEEE, Graduate Student Member (2020)


% \vspace{-8pt}
% \section*{Supervisor}
% \vspace{-8pt}
% \indent

% \hangindent=3em Dr. Fan Zhou, Asso. Prof., University of Electronic Science and Technology of China (UESTC), Chengdu, Sichuan 610054, China, email: fan.zhou@uestc.edu.cn

% \hangindent=3em Dr. Ting Zhong, Asso. Prof., University of Electronic Science and Technology of China (UESTC), Chengdu, Sichuan 610054, China, email: zhongting@uestc.edu.cn

% \vspace{-8pt}
% \section*{Student}
% \vspace{-8pt}
% \indent

\vspace{-8pt}
\section*{Publication}
\vspace{-4pt}
\indent

Note: * Corresponding, $\dagger$ Equal Contribution

Publication records can also be found in:

\begin{itemize}[leftmargin=4em, itemsep=0pt, topsep=0pt]
    \item Google Scholar: \url{https://scholar.google.com/citations?hl=en&user=ra0qyRQAAAAJ}
    \item publons: \url{https://publons.com/researcher/3651887/xovee-xu/}
    \item DBLP: \url{https://dblp.org/pid/261/9309.html}
\end{itemize}


\subsection*{\textnormal{\textit{Refereed Conference Articles}}}
\indent

\begin{enumerate}
    \item Fan Zhou, Xiuxiu Qi, \textbf{Xovee Xu}, Jiahao Wang*, Ting Zhong, and Goce Trajcevski. 2020. Meta-learned user preference for topic participation prediction. \textit{IEEE Global Communications Conference (GLOBECOM)}, Virtual Conference, Dec 7-11, 2020, 6 pages, doi:\href{https://doi.org/10.1109/GLOBECOM42002.2020.9322197}{10.1109/GLOBECOM42002.2020.9322197}
    \item Fan Zhou, Zijing Wen, Ting Zhong*, Goce Trajcevski, \textbf{Xovee Xu}, and Leyuan Liu. 2020. Unsupervised user identity linkage via graph neural networks. \textit{IEEE Global Communications Conference (GLOBECOM)}, Virtual Conference, Dec 7-11, 2020, 6 pages, doi:\href{https://doi.org/10.1109/GLOBECOM42002.2020.9322311}{10.1109/GLOBECOM42002.2020.9322311}
    \item Fan Zhou, Xin Jing, \textbf{Xovee Xu}, Ting Zhong, Goce Trajcevski, and Jin Wu*. 2020. Continual information cascade learning. \textit{IEEE Global Communications Conference (GLOBECOM)}, Virtual Conference, Dec 7-11, 2020, 6 pages, doi:\href{https://doi.org/10.1109/GLOBECOM42002.2020.9322124}{10.1109/GLOBECOM42002.2020.9322124}
    \item Fan Zhou, \textbf{Xovee Xu}, Kunpeng Zhang, Goce Trajcevski, and Ting Zhong*. 2020. Variational information diffusion for probabilistic cascade prediction. \textit{IEEE International Conference on Computer Communications (INFOCOM)}, Virtual Conference, Jul 6-9, 2020, pp. 1618-1627, doi:\href{https://doi.org/10.1109/INFOCOM41043.2020.9155349}{10.1109/INFOCOM41043.2020.9155349}
    \newline \textbf{\color{red}INFOCOM 2020 Student Conference Award}
\end{enumerate}

\subsection*{\textnormal{\textit{Refereed Journal Articles}}}

\begin{enumerate}[resume]
    \item Fan Zhou, \textbf{Xovee Xu}*, Goce Trajcevski, Kunpeng Zhang. 2021. A survey of information cascade analysis: Models, predictions, and recent advances. \textit{ACM Computing Surveys (CSUR)}, vol. 54, no. 2, article 27, 36 pages, Mar 2021, doi:\href{https://xovee.cn/html/paper-redirects/csur2021.html}{10.1145/3433000}, \href{https://arxiv.org/abs/2005.11041}{arXiv:2005.11041}
\end{enumerate}

\subsection*{\textnormal{\textit{Pre-prints \& On-going Articles}}}

\begin{enumerate}[resume]
    \item Fan Zhou, Liu Yu, \textbf{Xovee Xu}*, and Goce Trajcevski. 2021. Decoupling representation and regressor for long-tailed information cascade prediction. Under review, 2021
    \item \textbf{Xovee Xu}, Ting Zhong, Fan Zhou, Goce Trajcevski, and Xi Wu. 2021. Spatial-temporal contrasting for fine-grained urban flow super-resolution. Under review, 2021
    \item Fan Zhou, Pengyu Wang, \textbf{Xovee Xu}*, Wenxin Tai, and Goce Trajcevski. 2021. Contrastive trajectory learning for tour recommendation. Under review, 2021
    \item Fan Zhou, Ce Li, \textbf{Xovee Xu}, Xucheng Luo, and Ting Zhong. 2020. HGENA: A hyperbolic graph embedding approach for social network alignment. Under review, 2021
    \item \textbf{Xovee Xu}, Fan Zhou*, Kunpeng Zhang, and Siyuan Liu. 2020. CCGL: Contrastive cascade graph learning. Under review, 2020
    \item Fan Zhou, \textbf{Xovee Xu}*, Kunpeng Zhang, Siyuan Liu, and Goce Trajcevski. 2020. CasFlow: Exploring hierarchical structures and propagation uncertainty for cascade prediction. Under review, 2020
    \item Fan Zhou, Ce Li, \textbf{Xovee Xu}*, Zijing Wen, and Ting Zhong. 2020. Heterogeneous dynamical academic network for scientific impact propagation learning. Under review, 2021
\end{enumerate}

\vspace{-8pt}
\section*{Reference}
\vspace{-4pt}
\indent

\begin{tabular}{cc}
    \begin{minipage}[t]{.5\textwidth}
        \textbf{Dr. Fan Zhou} (M.S. Supervisor)\\
        Associate Professor\\
        University of Electronic Science and Technology of China\\Chengdu, Sichuan 610054, China\\
        fan.zhou@uestc.edu.cn
    \end{minipage}
    & 
    \begin{minipage}[t]{.4\textwidth}
        \textbf{Dr. Goce Trajcevski} (Collaborator)\\
        Harpole-Pentair Associate Professor\\
        Iowa State University\\
        Ames, IA 50011, USA\\
        gocet25@iastate.edu
    \end{minipage}
\end{tabular}



\end{document}
